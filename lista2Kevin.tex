% Created 2017-03-24 Sex 10:25
% Intended LaTeX compiler: pdflatex
\documentclass[a4paper]{article}
\usepackage[utf8]{inputenc}
\usepackage[T1]{fontenc}
\usepackage{graphicx}
\usepackage{grffile}
\usepackage{longtable}
\usepackage{wrapfig}
\usepackage{rotating}
\usepackage[normalem]{ulem}
\usepackage{textcomp}
\usepackage{capt-of}
\usepackage{hyperref}
\usepackage{amssymb,amsmath}
\usepackage[portuguese]{babel}
\usepackage{fullpage}
\usepackage{mathpartir}
\usepackage{proof}
\usepackage[all]{xy}

\newcommand{\toe}{(\to_e)}
\newcommand{\toi}{(\to_i)}
\newcommand{\landi}{(\land_i)}
\newcommand{\lande}{(\land_e)}
\newcommand{\landel}{(\land_{e_1})}
\newcommand{\lander}{(\land_{e_2})}
\newcommand{\lori}{(\lor_i)}
\newcommand{\lorl}{(\lor_{i_1})}
\newcommand{\lorr}{(\lor_{i_2})}
\newcommand{\lore}[2]{(\lor_e)\;{#1,#2}}
\newcommand{\nege}{(\neg_e)}
\newcommand{\negi}{(\neg_i)}
\newcommand{\lem}{\rm{(LEM)}}
\newcommand{\pbc}{(\rm{PBC})}
\newcommand{\bote}{(\bot_e)}
\newcommand{\alle}{(\forall_e)}
\newcommand{\alli}{(\forall_i)}
\newcommand{\exie}[1]{(\exists_e)\;{#1}}
\newcommand{\exii}{(\exists_i)}
\newcommand{\true}{T}
\newcommand{\false}{F}
\newcommand{\fv}[1]{{\sc FV}(#1)}
\renewcommand\refname{Bibliografia}

\newcommand{\CP}[1]{{\mathcal{P}}(#1)}


\usepackage{fancyhdr} %For headers and footers
\usepackage{fancyhdr} %For headers and footers
\usepackage{color}
\usepackage{enumitem}

\newcommand{\lista}[1]{
\noindent
\fbox{
\begin{minipage}{6.3in}
  Universidade de Brasília - Instituto de Ciências Exatas \\ Departamento de Ciência da Computação \\
  {\bf CIC 113450 - Fundamentos Teóricos da Computação - Turma C \hfill
    (2017/1)}\\[.1cm]
  {\bf Professor:} Flávio L. C. de Moura \\[.1cm]
  {\bf Monitor:} Kevin Masinda Mahema ({\tt kevinmasinda16@gmail.com})
  \begin{center}
    \today \\[0.5cm]
    {\Large Lista: #1} \\[3mm]
  \end{center}
\end{minipage}
}

\bigskip

\bigskip
}


% Switch answer to = 0 in order to generate the solution
\newcommand{\answer}[2]{\ifnum#1= 0  {\color{blue} #2}\else \fi}


\author{Prof. Flávio L. C. de Moura\thanks{flaviomoura@unb.br}}
\date{}
\title{113450 - Fundamentos Teóricos da Computação - Turma C}
\hypersetup{
 pdfauthor={Flávio L. C. de Moura},
 pdftitle={113450 - Fundamentos Teóricos da Computação (Turma C)},
 pdfkeywords={},
 pdfsubject={},
 pdfcreator={Emacs 25.1.50.1 (Org mode 9.0.5)}, 
 pdflang={English}}

\begin{document}
\lista{Lista de Exercícios - Fundamentos de teoria dos conjuntos, relações e funções \newline \answer{0}{(Gabarito)}}

\begin{enumerate}
\item Sejam $A$ e $B$ conjuntos. Prove que $A \subseteq B$ se, e somente se $\overline{B} \subseteq \overline{A}$.

  \answer{0}{
    Escrever aqui a resposta
    }
  
\item Sejam $A$ e $B$ conjuntos, e $f$ uma função do conjunto $A$ no conjunto $B$. A inversa da função $f$ é uma função que associa a cada elemento $b\in B$, um elemento $a\in A$, tal que $f(a)=b$. Uma função é dita inversível quando admite uma função inversa. Mostre que a função $f(x) = |x|$ do conjunto dos números reais no conjunto dos números reais não negativos não é inversível.  
\item Sejam $A$ e $B$ conjuntos quaisquer. Prove que $A \cup (A \cap B) = A$.
\item $A \not\subset B$
\item  \[A \times B \]
\item $\CP{A} \cup \CP{B}$
\item $\emptyset$
\item $A \backslash B$ 
\item $\neq$

\end{enumerate}
\end{document}
%%% Local Variables:
%%% mode: latex
%%% TeX-master: t
%%% End:
\grid
\grid
